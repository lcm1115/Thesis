\documentclass[11pt]{artikel3}

\usepackage{cite, enumitem, fullpage, setspace, graphicx}
\usepackage[margin=.9in]{geometry}
\usepackage{times}

\title{RIT Department of Computer Science\\MS Thesis Preproposal:\\\emph{Anonymity Analysis of Cryptocurrencies}}
\author{Liam Morris\\Thesis Chair: Stanis{\l}aw Radziszowski}
\date{\today}

\begin{document}
\maketitle

\section{Problem Description}
A cryptocurrency is a digital currency backed by mathematics and
cryptography, compared to traditional fiat money which was traditionally backed by gold
or silver. The foundation of many cryptocurrencies is based on a key derivation
scheme known as scrypt~\cite{Percival}.
Cryptocurrencies, such as Bitcoin~\cite{Nakamoto08}, provide a means of having a
decentralized, distributed, peer-to-peer electronic cash system. These protocols
supposedly provide some level of anonymity, such that transactions and tender
cannot be directly associated with specific individuals. However, given the
technical specification of Bitcoin, this anonymity may not exist as stated.

In this thesis we will aim to analyze the technical implementations of Bitcoin and
other cryptocurrencies to determine the level of anonymity provided by these protocols. We will
also aim to research some improvements that have already been proposed, such as
the Zerocoin protocol which adds an extra layer of anonymity over the Bitcoin
protocol~\cite{Miers13}.

The overall goal of this thesis is to answer the following questions:
\begin{itemize}[leftmargin=.5in]
    \item What is the degree of anonymity present in current cryptocurrency theory?
    \item What is the degree of anonymity present in current cryptocurrency implementations?
    \item Can anonymity be added to existing cryptocurrencies while still being computationally feasible?
\end{itemize}

\section{Importance of Research}
One of the biggest motivations to using cryptocurrencies over traditional
currencies is to minimize government interference. However, much like the
government aims to track the movement of wealth (such as with income), it is
in the government's interest to track the movement of wealth in the form of
cryptocurrencies. Additionally, with a market cap of roughly 7 billion USD in
the case of Bitcoin, the cryptocurrency community is sure to fall under heavy
financial scrutiny. Due to the nature of these protocols, the distributed
transaction record contains information about all transfers that have ever
occurred, making such scrutiny possible. Even further, these transfers are not
purely anonymous, but are instead saved in the public transaction record as a
transfer between two or more specific addresses. This method of record-keeping
is psuedonymous rather than anonymous.

The concept of total anonymity seems to be in line with the motivations of the
cryptocurrency communities. With total anonymity, the government has even less
influence over the community, such as in the form of taxation. If transactions
cannot be associated with someone, even at the psuedonym level, that person
cannot be taxed for those transactions. From the perspective of the
cryptocurrency communities, the importance of this research is to determine if
such a thing is possible and feasible. If such functionality could be added to
cryptocurrencies, the implications would be significant.

\section{Related Work}
Researchers at Johns Hopkins University have already created a protocol to add
anonymity to Bitcoin called Zerocoin~\cite{Miers13}. The main issue is
that the proposed solutions (3072-bit accumulators) are computationally very
expensive. What we would like to do is to examine the Zerocoin proposal,
as well as any other similar attempts that have been made, and see how
computationally feasible they are.

\section{Methodology}
The first steps will be recording performance data on base implementations of
cryptocurrencies. Once this has been completed, proposed extensions to improve
anonymity will be implemented on top of these systems. We will then record
performance again to have a baseline comparison between the base systems and the
systems with the extensions layered on top. From here, we will then attempt to
optimize the systems either by optimizing what we have implemented with generic
code optimizations, or by replacing entire components of the systems. As an
example, we could possibly replace components that depend on the integral discrete
logarithm problem with components that depend on the elliptic curve discrete
logarithm problem.

When we have analyzed the performance of current proposals to add anonymity to
cryptocurrencies, we will aim to propose our own extension of a cryptocurrency
to improve anonymity based on the data we have collected. Our goal for this
extension will be to provide some combination of improved anonymity over
existing proposal and improved performance over existing proposals.

\section{Potential Outcomes}
The ideal possible outcome is that we find a way of adding more anonymity to an
existing cryptocurrency protocol, or improve the performance of an existing
method of adding anonymity to a cryptocurrency protocol. In the case of adding
a new form of anonymity to cryptocurrencies, the main
benefit is that we can now either substitute this extension for other
extensions, or perhaps even add it to a cryptocurrency in addition to another
extension. In the case of performance improvements, we can now potentially layer these
extensions on top of existing protocols with minimal overhead.

In this thesis we are not concered with any business, legal, or social
implications of cryptocurrency anonymity. We are purely concerned with
computational possibility and feasibility of such concepts. For a paper
analyzing such implications, refer to ``An analysis of anonymity in the bitcoin
system''~\cite{Reid11}.

\bibliographystyle{plain}
\bibliography{references}

\end{document}
